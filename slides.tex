\PassOptionsToPackage{usenames,dvipsnames,xcolor=x11names,table}{xcolor}
\documentclass[xelatex,aspectratio=169]{beamer}

\usepackage{xifthen,multicol,textcomp,graphicx,talk,minted,epigraph}
\usepackage{fontspec}
\usepackage{tikz,amsmath}
\usepackage[binary-units=true]{siunitx}
\usepackage{array, booktabs}
%\usepackage{graphicx}
%\usepackage[x11names]{xcolor}
%\usepackage{colortbl}
\usepackage{caption}
\DeclareCaptionFont{blue}{\color{Blue}}
\newcommand{\foo}{\color{Blue}\makebox[0pt]{\textbullet}\hskip-0.5pt\vrule width 1pt\hspace{\labelsep}}
\usetikzlibrary{arrows,decorations.pathreplacing,decorations.text}
\newcommand*{\tikzmark}[1]{\tikz[overlay, remember picture] \coordinate ({#1});}

\usetheme{FOSDEM}
\usecolortheme{whale}
%\beamertemplatenavigationsymbolsempty

\title[]{The Case for \interface}
\subtitle{FOSDEM'18}
\author[]{%
	Sam Whited\\*%
	{\tiny JID: \texttt{\href{xmpp:sam@samwhited.com}{sam@samwhited.com}}}%
}
\date{2017--02--03}
\titlegraphic{\includegraphics[width=20mm]{images/cf-logo-v-rgb.png}\hspace*{.25\textwidth}\includegraphics[width=20mm]{images/xmpp.png}}

% Define a font family that supports IPA characters.
\newfontfamily\ipa{Charis SIL}

\begin{document}

\section{intro}

\begin{frame}
	\maketitle
\end{frame}

\newfontfamily\blackout{Blackout}
\begin{frame}
	\begin{center}
		\LARGE\interface
	\end{center}

\only<2>{%
	\color{Red}\blackout
	\begin{tikzpicture}[remember picture, overlay]
		\node [shift={(-3cm,2.3cm)}] at (current page.south east) {%
			\begin{tikzpicture}[decoration={text effects along path,
						text={Fear! Thrills! Horror!},
						text effects/.cd,
						path from text, path from text angle=60,
						character count=\i, character total=\n,
						characters={text along path, scale=\i/\n+1}}]
							\path[decorate] (0,2) .. controls ++(2,0) and ++(-2,0) .. (6,4);
			\end{tikzpicture}
		};
	\end{tikzpicture}
}
\end{frame}

% \section{Abstract}
% \begin{frame}
% 	\frametitle{Abstract}
% 	The empty interface (\interface) is one of the most interesting, powerful, and
% 	easily abused features of Go, but there are use cases for which it is uniquely
% 	and excellently suited.
% 	In this talk for the Go devroom, I will propose a set of rules which can be
% 	used to see if your use of \interface\ will be the elegant and simple API that
% 	\interface\ promised, or merely lead to a maintainability nightmare down the
% 	road.
% \end{frame}

\section{assumptions}
\begin{frame}
	\ttfamily%
	The following requirements keywords as used in this document are to be
	interpreted as described in RFC 2119: "MUST", "SHALL", "REQUIRED"; "MUST NOT",
	"SHALL NOT"; "SHOULD", "RECOMMENDED"; "SHOULD NOT", "NOT RECOMMENDED"; "MAY",
	"OPTIONAL".
	\begin{tikzpicture}[remember picture,overlay]
		\node [xshift=-0.25\textwidth,yshift=6em] at (current page.south east)
		{\includegraphics[width=7em]{images/gopherhorse.png}};
	\end{tikzpicture}
\end{frame}

\begin{frame}
	\only<1>{\setbeamercolor{item}{fg=Violet}}
	\only<2>{\setbeamercolor{item}{fg=CheetoOrange}}
	\only<3>{\setbeamercolor{item}{fg=Mauve}}
	\only<4>{\setbeamercolor{item}{fg=Slate}}
	\only<5->{\setbeamercolor{item}{fg=LimeGreen}}
	\begin{enumerate}
		\item<1> In Go, interfaces should describe behavior, not data
		\item<2> \interface\ is easy to abuse (and thus, is abused; widely and often)
		\item<3> \interface\ is code for ``dynamic typing"
		\item<4> If you can describe your behavior with a more specific type, you should
		\item<5-> Heavy use of reflection leads to difficult to maintain code
	\end{enumerate}
	\begin{center}
		\only<6->{\LARGE \pkg{encoding/\{json,xml\}}}
	\end{center}
	\only<7->{\vspace*{\fill}}
	\begin{flushright}
		\only<7>{Q.E.D.}
	\end{flushright}
	\vspace*{\fill}
	\begin{tikzpicture}[remember picture,overlay]
		\only<1->{
		\node [xshift=0.25\textwidth,yshift=2.5em] at (current page.south west)
		{\includegraphics[width=20mm]{images/gopher.png}};}
	\end{tikzpicture}
\end{frame}

\section{\pkg{encoding/xml}}
\frame{\LARGE \sectionpage}

\begin{frame}
	\only<1,3->{\inputminted{go}{example1.go}}
	\only<2>{\epigraph{Empty interface says nothing}{Rob Pike, Gopherfest 2015}}
	\only<4->{\begin{enumerate}}
		\only<4->{\setbeamercolor{item}{fg=Yellow}\item{1\hspace*{10.5em}→ <int>1</int>}}
		\only<5->{\setbeamercolor{item}{fg=Red}\item{``To sit in solemn silence" → <string>To sit in solemn silence</string>}}
		\only<6->{\setbeamercolor{item}{fg=MidGreen}\item{\type{xml.Marshaler}\hspace*{4.2em}→ \method{T.MarshalXML(e, start)}}}
		\only<7->{\setbeamercolor{item}{fg=Slate}\item{struct\{ Name string \}\hspace*{1.8em}→ ???\only<8->{\ (reflection!)}}}
	\only<4->{\end{enumerate}}
\end{frame}

\begin{frame}
	\Large
	``When the \emph{producer} of some data does not care about the type, but the
	\emph{consumer} does, the library becomes difficult to maintain."
\end{frame}

\section{\pkg{Rule \textnumero 1}}
\begin{frame}
	\Large
	\strong{Rule \textnumero 1}\\
	The producer of the \interface\ must also be the consumer of the \interface.
\end{frame}

\section{\pkg{context}}
\frame{\LARGE \sectionpage}

\begin{frame}
	\inputminted{go}{example2.go}
\end{frame}

\begin{frame}
	% TODO: better quotation formatting?
	\texttt{\color{MidGreen}// }\textit{%
		Package context defines the Context type, which carries deadlines,
		cancelation}\\
	\texttt{\color{MidGreen}// }\textit{%
		signals, and other
		\only<1>{request-scoped values}\only<2>{\strong{request-scoped values}}
		across API boundaries and
		between}\\
	\texttt{\color{MidGreen}// }\textit{processes.}
\end{frame}

\begin{frame}
	\begin{enumerate}
		\only<1->{\setbeamercolor{item}{fg=LimeGreen}\item{Session ID}}
		\only<2->{\setbeamercolor{item}{fg=Rosie}\item{Request ID}}
		\only<3->{\setbeamercolor{item}{fg=Cyan}\item{Trace ID}}
	\end{enumerate}
\end{frame}

\begin{frame}
	\begin{multicols}{2}
		\inputminted{go}{example3.go}
		\columnbreak
		\inputminted{go}{example4.go}
	\end{multicols}
\end{frame}

\begin{frame}
		\inputminted{go}{example5.go}
\end{frame}

\section{\pkg{Rule \textnumero 2}}
\begin{frame}
	\Large
	\strong{Rule \textnumero 2}\\
	\interface\ should not cross package boundaries.
\end{frame}

\section{\pkg{sasl}}
\frame{\LARGE \sectionpage}

\begin{frame}
	\inputminted{go}{example6.go}
\end{frame}

% \begin{frame}
% 	\setlength{\lineskip}{0pt}
% 	┌────────────┐ ┌───────────┐
% 	│ Negotiator │ │ Mechanism │
% 	└─────┬──────┘ └─────┬─────┘
%         │              │
%         │              │
%         │              │
%         │              │
% 	┌─────┴──────┐ ┌─────┴─────┐
% 	│ Negotiator │ │ Mechanism │
% 	└────────────┘ └───────────┘
% \end{frame}

\begin{frame}
	\inputminted{go}{example7.go}
\end{frame}

\section{\pkg{Rule \textnumero 3}}
\begin{frame}
	\Large
	\strong{Rule \textnumero 3}\\
	You must always be able to assert the type of the \interface.
\end{frame}

\section{fin}

\newfontfamily\calluna{Calluna}
\begin{frame}
	\Large
	\calluna
	\begin{figure}
	{\color{AshGray}%
		\tikzmark{begin}sam\tikzmark{atsep}@%
		\tikzmark{domainpart}\textcolor{Cyan}{SamWhited}\tikzmark{tld}.com\tikzmark{end}%
	}

		\tikz[overlay,remember picture] {
			\draw[decorate,decoration={brace,raise=5mm,amplitude=15pt}] (begin.north
			west) -- node [above=1.75em] {\small email \& jid} (end.north east) ;
			\draw[decorate,decoration={brace,raise=2mm,amplitude=5pt,mirror}]
			(begin.south west) -- node[below of=begin, below=-1.35em] {\tiny me} (atsep.south west) ;
			\draw[decorate,decoration={brace,raise=2mm,amplitude=5pt,mirror}]
			(domainpart.south west) -- node[below of=begin, below=-1.35em] {\tiny twitter
			\addfontfeature{Variant=1}{\&} freenode } (tld.south west) ;
			\draw[decorate,decoration={brace,raise=6mm,amplitude=10pt,mirror}]
			(domainpart.south west) -- node[below of=begin, below=-.125em] {\tiny blog} (end.south west) ;
		}
	\end{figure}
\end{frame}

\end{document}
