\documentclass{article}
\usepackage{listings,talk,minted}

\title{The Case for \interface}
\author{Sam Whited}
\date{February 02, 2018}

\begin{document}
\maketitle
\abstract{%
	The empty interface (\interface) is one of the most interesting, powerful,
	and easily abused features of Go, but there are use cases for which it is
	uniquely and excellently suited.
	In this talk for the Go devroom, I will propose a set of rules which can be
	used to see if your use of \interface\ will be the elegant and simple API
	that \interface\ promised, or merely lead to a maintainability nightmare down
	the road.
}

\section{Introduction}

In this crowd I suspect I don't have to start with an explanation of \interface\
or its dangers, but let me briefly lay out a few assumptions that will be
essential going forward:

\begin{itemize}
	\item{In Go, interfaces should describe behavior, not data}
	\item{\interface\ is easy to abuse (and thus, is abused; widely and often)}
	\item{\interface\ is code for ``dynamic typing"}
\end{itemize}

That is to say, whenever I say ``\interface", you should hear ``dynamic typing".
Let us also assume that it's widely acknowledged that if you can describe your
behavior with a more specific type, you should.
Finally, let us take it as a given that heavy use of reflection leads to
difficult to maintain code.
If you don't agree, please attempt to make changes to one of the
\pkg{encoding/*} packages and then get back to me.

\section{Library Maintainability}

Let's start by talking about maintainability by taking an example from the
standard library, in fact, from \pkg{encoding/xml}:

\begin{lstlisting}
	// Marshal returns the XML encoding of v.
	func Marshal(v interface{}) ([]byte, error)
\end{lstlisting}

We can say that this function ``consumes" a dynamically typed value.
Because, as Rob Pike famously said, ``empty interface says nothing", it must
rely on a mix of type switching and reflection to determind how the value should
be marshaled into XML.
For primitive types and types defined in the \pkg{encoding/xml} package this
works well enough, we type switch and the package can perform some pre-defined
behavior.
For example, if the type is an \method{xml.Marshaler} it can call its
\method{MarshalXML} method, or if it is a primitive type the logic for
marshaling the type is already contained in the package.
However, for types that are defined outside of the \pkg{encoding/xml} package
things are more complicated. Type erasure means we have no information about
a user defined struct; no methods to call, no known fields, etc.
Since no information is known about the type we either must limit the usability
of the package by not allowing user defined types or gather the information we
need using reflection, leading to maintainability problems.
As I said before, if you've ever tried to modify \pkg{encoding/xml} or
\pkg{encoding/json}, you'll know what I'm talking about: part of your code base
is heavily reflection based, and part of your code is normal, idiomatic Go.
It becomes difficult to reconcile the two distinct code bases.

The \method{MarshalXML} API is simple, even elegant, from the users perspective.
The user doesn't know or care about the type of data: they put arbitrary data in
and get some serialized representation of that data out.
This contract is reflected with appropriate types in the function signature.
However, from the library developers perspective things became very hard to
maintain because the library \emph{does} need to care about the type of the data
being put in.
When the \emph{producer} of some data does not care about the type, but the
\emph{consumer} does, the library becomes difficult to maintain.

Rephrased, this becomes our first rule for using \interface:
``The producer of the \interface must also be the consumer of the \interface."

\section{Application Maintainability}

Let's talk about the \pkg{context} package.

\inputminted{go}{example2.go}

\pkg{context} does not have the same problems as \pkg{encoding/xml}.
Context is commonly used to write HTTP middleware, and when doing so, a library
using context generally defines the key and value, for instance in a
"WithSessionID" middleware, and generally also provides a way to consume that
value ("GetSessionID").
This API means that the context package does not need to make heavy use of
reflection, and the package is a study in simplicity.
If you're learning to write idiomatic Go and haven't looked at the source of
\pkg{context}, I highly recommend it.

However, \pkg{context} has a different problem: a value stored in an empty
interface in context may be modified by many different producers, and consumed
by many different consumers.
This means that if I add middleware to write a request ID into my context, and
then someone else adds my middleware to an existing pipeline later my request
ID may end up being silently masked further down the context tree.
This, and the fact that context trees may have many values appended to them,
leads to a great deal of complexity at the application layer.
The producer of our \interface\ is also the consumer, but it may produce or
consume values multiple times from many locations.

\end{document}
