\documentclass{article}
\usepackage{listings,talk,minted,multicol}

\title{The Case for \interface}
\author{Sam Whited}
\date{February 02, 2018}

\begin{document}
\maketitle
\abstract{%
	The empty interface (\interface) is one of the most interesting, powerful,
	and easily abused features of Go, but there are use cases for which it is
	uniquely and excellently suited.
	In this talk for the Go devroom, I will propose a set of rules which can be
	used to see if your use of \interface\ will be the elegant and simple API
	that \interface\ promised, or merely lead to a maintainability nightmare down
	the road.
}

\section{Introduction}

In this crowd I suspect I don't have to start with an explanation of \interface\
or its dangers, but let me briefly lay out a few assumptions that will be
essential going forward:

\begin{itemize}
	\item{In Go, interfaces should describe behavior, not data}
	\item{\interface\ is easy to abuse (and thus, is abused; widely and often)}
	\item{\interface\ is code for ``dynamic typing"}
\end{itemize}

That is to say, whenever I say ``\interface", you should hear ``dynamic typing".
Let us also assume that it's widely acknowledged that if you can describe your
behavior with a more specific type, you should.
Finally, let us take it as a given that heavy use of reflection leads to
difficult to maintain code.
If you don't agree, please attempt to make changes to one of the
\pkg{encoding/*} packages and then get back to me.

\section{Library Maintainability}

Let's start by talking about maintainability by taking an example from the
standard library, in fact, from \pkg{encoding/xml}:

\begin{lstlisting}
	// Marshal returns the XML encoding of v.
	func Marshal(v interface{}) ([]byte, error)
\end{lstlisting}

We can say that this function ``consumes" a dynamically typed value.
Because, as Rob Pike famously said, ``empty interface says nothing", it must
rely on a mix of type switching and reflection to determind how the value should
be marshaled into XML.
For primitive types and types defined in the \pkg{encoding/xml} package this
works well enough, we type switch and the package can perform some pre-defined
behavior.
For example, if the type is an \method{xml.Marshaler} it can call its
\method{MarshalXML} method, or if it is a primitive type the logic for
marshaling the type is already contained in the package.
However, for types that are defined outside of the \pkg{encoding/xml} package
things are more complicated. Type erasure means we have no information about
a user defined struct; no methods to call, no known fields, etc.
Since no information is known about the type we either must limit the usability
of the package by not allowing user defined types or gather the information we
need using reflection, leading to maintainability problems.
As I said before, if you've ever tried to modify \pkg{encoding/xml} or
\pkg{encoding/json}, you'll know what I'm talking about: part of your code base
is heavily reflection based, and part of your code is normal, idiomatic Go.
It becomes difficult to reconcile the two distinct code bases.

The \method{MarshalXML} API is simple, even elegant, from the users perspective.
The user doesn't know or care about the type of data: they put arbitrary data in
and get some serialized representation of that data out.
This contract is reflected with appropriate types in the function signature.
However, from the library developers perspective things became very hard to
maintain because the library \emph{does} need to care about the type of the data
being put in.
When the \emph{producer} of some data does not care about the type, but the
\emph{consumer} does, the library becomes difficult to maintain.

Rephrased, this becomes our first rule for using \interface:
``The producer of the \interface must also be the consumer of the \interface."

\section{Application Maintainability}

Let's talk about the \pkg{context} package.

\inputminted{go}{example2.go}

\pkg{context} does not have the same problems as \pkg{encoding/xml}.
Context is commonly used to write HTTP middleware, and when doing so, a library
using context generally defines the key and value, for instance in a
``WithSessionID" middleware, and generally also provides a way to consume that
value (``GetSessionID").
This API means that the context package does not need to make heavy use of
reflection, and the package is a study in simplicity.
If you're learning to write idiomatic Go and haven't looked at the source of
\pkg{context}, I highly recommend it.

However, \pkg{context} has a different problem.
The documentation for the \pkg{context} package says:
``Package context defines the Context type, which carries deadlines, cancelation
signals, and other request-scoped values across API boundaries and between
processes."
Let's stop for a moment to think about ``request scoped values" means.
I suspect the intent was for that short list to include things like:
\begin{enumerate}
	\item Request ID
	\item Session ID
	\item Trace ID
\end{enumerate}
And that's about it really: anything that should be accessable from anywhere
that can reference the request and ends with ``ID".

Unfortunately, I suspect we've all seen the horror that
is a developer who's used to certain Java web frameworks discovering Go's
``dependency injection package":

\begin{multicols}{2}
	\inputminted{go}{example3.go}
\end{multicols}

This is an understandable mistake, and soon corrected.
Side channel APIs that are, effectively, stringly-typed quickly become a
maintainability problem when scaling your team to a size greater than one.
So why do developers abuse context as a side-channel API for dependency
injection so often?

Certainly it can be partially attributed to new Go users coming from other
languages where this is an accepted pattern, but there is also a simpler answer:
``because they can."
As a rule if something can be abused, it will.
And in this case the abuse is made possible by context's use of the empty
interface.
Take, for example the logging middleware mentioned earlier:

\inputminted{go}{example5.go}

This is broken \emph{because} of our first rule: the producer of the value in
the \interface\ is also the consumer, and therefore they can use the \interface\
to create a side channel API, passing arbitrary values between
\method{ServeHTTP} calls.
We've traded unmaintainability in our library for unmaintainability in our
user's library or application by allowing them to produce and consume
\interface\ typed values.
To fix this, we need another rule:
``\interface\ should not cross package boundaries."
If you must use \interface\ in your library, ensure that it is an implementation
detail and not leaked in a usable way by any exported function or method.

Finally, let's look at an example that follows these rules:

\inputminted{go}{example6.go}

The Simple Authentication and Security Layer (SASL) is an authentication
framework defined by RFC 4422.
You probably use it every time you access your email, auth to Freenode, etc.
The specifcs don't matter, except that SASL supports many authentication
mechanisms; these are things like plain text, SCRAM, OAUTH2, etc. and that,
because these mechanisms may require multiple round trips to the server there is
generally a state machine, the ``\struct{Negotiator}" in this API, that controls
the auth flow, ensures that steps cannot be repeated, and generally enforces
security constraints.

As you can see, the \struct{Mechanism} makes use of \interface.
In each step of the state machine the selected mechanism's `\method{Next}'
function is called.
\struct{Mechanism}, for reasons that don't matter here, must remain stateless.
Unfortunately, mechanisms also may need data from previous steps, and this state
must be stored somewhere.
Since mechanisms are used by a state machine, which is (obviously) already
stateful, it makes sense to store that state as part of the \struct{Negotiator}.
SASL is also a very flexible framework, so we don't necessarily know what that
data will be: sounds like a job for \interface.
Each time the `\method{Next}' function is called it optionally returns a value
that will be needed in a future step.
The negotiator then stores this state, and the next time it calls \method{Next}
it passes in the stored value.
In practice, a stateless \struct{Mechanism}'s \method{Next} functions end up
looking something like this:

\pagebreak
\inputminted{go}{example7.go}

The mechanism is not only the producer of the value in the interface (the
``cache" return parmeter), it is also the consumer of the value (the ``data"
argument), so it meets our first requirement.
This \interface\ is also an internal implementation detail: it never leaks
outside of the package.
Together, this means that \struct{Mechanism}'s remain simple: reflection is not
necessary, and the \interface\ is not a part of a public API, so it can't be
used to expose a side-channel for smuggling data into a system.

\end{document}
